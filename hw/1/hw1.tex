\documentclass[12pt, a4paper]{article}
\usepackage[english]{babel}
\usepackage[utf8]{inputenc}
\usepackage[T1]{fontenc}
\usepackage{enumitem}
\usepackage{amsmath}
\usepackage{hyperref}
\usepackage{graphicx}
\usepackage{animate}
\usepackage{tikz}
\usetikzlibrary{lindenmayersystems}

\pgfdeclarelindenmayersystem{A}{%
 \symbol{F}{\pgflsystemstep=0.6\pgflsystemstep\pgflsystemdrawforward}
 \rule{A->F[+A][-A]}
}

\title{AML -- Homework Assignment 1}
\author{Nace Zavrtanik}


\begin{document}

\maketitle


\noindent NOTE: While tackling the homework assignment, I have relied on a function I wrote when solving exercise classes:
\verb|compare_models_cross_validation|. For more information on \verb|compare_models_cross_validation| refer
to its documentation in the code.

%%%%%%%%%%%%%%%%%%%%%%%%% NALOGA 1 %%%%%%%%%%%%%%%%%%%%%%%%%
\section{Method Selection and Hyperparameter Optimisation}


\subsection{Manual approach}
I used \verb|compare_models_cross_validation| to gauge the effectiveness of the following models on the given dataset:
\begin{itemize}
\item \verb|sklearn.ensemle.RandomForestClassifier|,
\item \verb|sklearn.tree.DecisionTreeClassifier|,
\item \verb|sklearn.naive_bayes.GaussianNB|,
\item \verb|sklearn.neighbors.KNeighborsClassifier|.
\end{itemize}
\begin{center}
\includegraphics[scale=0.8]{comparison_models.png}
\end{center}
As \verb|sklearn.ensemle.RandomForestClassifier| had the highest ROC AUC score, I chose to further tune the hyperparameters
for that particular model.

I performed the manual hyperparameter optimisation in two different ways, the first relying mostly on my own
implementation, and the second using the functionality of \verb|sklearn.model_selection.GridSearchCV|. Both
optimisation methods were used on the same predefined hyperparameter grid, containing:
\begin{itemize}
\item hyperparameter values of different magnitudes,
\item default hyperparameters values.
\end{itemize}

\begin{verbatim}
HYPERPARAMETER OPTIMISATION -- Manual
n_estimators ........................... 50
max_depth .............................. 32
min_samples_split ...................... 5
roc_auc ................................ 0.9471312515866971

HYPERPARAMETER OPTIMISATION -- GridSeachCV
max_depth .............................. 14
min_samples_split ...................... 2
n_estimators ........................... 124
roc_auc ................................ 0.9417999492256918
\end{verbatim}

\subsection{Automated approach}
For the automated hyperparameter optimisation with \verb|hyperopt|, I chose the same models as above, picked the hyperparameter
space based on the same criteria as above, and chose the random forest model to tune more hyperparameters for, as it had the highest
ROC AUC score in the case above. (For the particular choice of the hyperparameter set refer to the code.) It is perhaps worth mentioning
that the hyperparameter \verb|var_smoothing| for the Bayesian model is the only continuous hyperparameter and is hence specified
using the \verb|hyperopt.hp.loguniform| method rather than the method for discrete values, \verb|hyperopt.hp.choice|. Again, the
random forest classifier performed best.

\begin{verbatim}
HYPERPARAMETER OPTIMISATION -- hyperopt
max_depth .............................. 32
min_samples_split ...................... 402
n_estimators ........................... 4
roc_auc ................................ 0.8613861386138613
\end{verbatim}

\begin{center}
\animategraphics[autoplay, loop, width=14cm, controls=all]{1}{frame}{1}{12}
\end{center}

The graph above displays the trials from the automated optimisation process with regard to the loss function. Lower loss
values indicate a higher ROC AUC score and hence a better performance. Models are differentiated by the color of the dots, while
hyperparameter values for particular trials appear on hover. A strict hierarchy of models is immediately visible, and it is in line with
the preliminary results by \verb|compare_models_cross_validation| above. It is clear that model selection is of far greater importance
for the quality of the predictions than the choice of hyperparameter values.

Notably, out of the three optimisation approaches, the automated approach gave the worst results. I have been unable to make the
optimisation with \verb|hyperopt| repeatable by fixing a random seed, so I ran the optimisation method several times. The ROC AUC
score presented above is one of the worst this approach has produced, some other being:
\begin{verbatim}
HYPERPARAMETER OPTIMISATION -- hyperopt
roc_auc ................................ 0.9087331810104088
roc_auc ................................ 0.925996445798426
\end{verbatim}

%%%%%%%%%%%%%%%%%%%%%%%%% NALOGA 2 %%%%%%%%%%%%%%%%%%%%%%%%%
\section{Meta-learning}




























\end{document}
